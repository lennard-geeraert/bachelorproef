%==============================================================================
% Sjabloon onderzoeksvoorstel bachelorproef
%==============================================================================
% Gebaseerd op LaTeX-sjabloon ‘Stylish Article’ (zie voorstel.cls)
% Auteur: Jens Buysse, Bert Van Vreckem
%
% Compileren in TeXstudio:
%
% - Zorg dat Biber de bibliografie compileert (en niet Biblatex)
%   Options > Configure > Build > Default Bibliography Tool: "txs:///biber"
% - F5 om te compileren en het resultaat te bekijken.
% - Als de bibliografie niet zichtbaar is, probeer dan F5 - F8 - F5
%   Met F8 compileer je de bibliografie apart.
%
% Als je JabRef gebruikt voor het bijhouden van de bibliografie, zorg dan
% dat je in ``biblatex''-modus opslaat: File > Switch to BibLaTeX mode.

\documentclass{voorstel}

\usepackage{lipsum}

%------------------------------------------------------------------------------
% Metadata over het voorstel
%------------------------------------------------------------------------------

%---------- Titel & auteur ----------------------------------------------------

% TODO: geef werktitel van je eigen voorstel op
\PaperTitle{Titel voorstel}
\PaperType{Onderzoeksvoorstel Bachelorproef 2018-2019} % Type document

% TODO: vul je eigen naam in als auteur, geef ook je emailadres mee!
\Authors{Steven Stevens\textsuperscript{1}} % Authors
\CoPromotor{Piet Pieters\textsuperscript{2} (Bedrijfsnaam)}
\affiliation{\textbf{Contact:}
  \textsuperscript{1} \href{mailto:steven.stevens.u1234@student.hogent.be}{steven.stevens.u1234@student.hogent.be};
  \textsuperscript{2} \href{mailto:piet.pieters@acme.be}{piet.pieters@acme.be};
}

%---------- Abstract ----------------------------------------------------------

\Abstract{Hier schrijf je de samenvatting van je voorstel, als een doorlopende tekst van één paragraaf. Wat hier zeker in moet vermeld worden: \textbf{Context} (Waarom is dit werk belangrijk?); \textbf{Nood} (Waarom moet dit onderzocht worden?); \textbf{Taak} (Wat ga je (ongeveer) doen?); \textbf{Object} (Wat staat in dit document geschreven?); \textbf{Resultaat} (Wat verwacht je van je onderzoek?); \textbf{Conclusie} (Wat verwacht je van van de conclusies?); \textbf{Perspectief} (Wat zegt de toekomst voor dit werk?).

Bij de sleutelwoorden geef je het onderzoeksdomein, samen met andere sleutelwoorden die je werk beschrijven.

Vergeet ook niet je co-promotor op te geven.
}

%---------- Onderzoeksdomein en sleutelwoorden --------------------------------
% TODO: Sleutelwoorden:
%
% Het eerste sleutelwoord beschrijft het onderzoeksdomein. Je kan kiezen uit
% deze lijst:
%
% - Mobiele applicatieontwikkeling
% - Webapplicatieontwikkeling
% - Applicatieontwikkeling (andere)
% - Systeembeheer
% - Netwerkbeheer
% - Mainframe
% - E-business
% - Databanken en big data
% - Machineleertechnieken en kunstmatige intelligentie
% - Andere (specifieer)
%
% De andere sleutelwoorden zijn vrij te kiezen

\Keywords{Onderzoeksdomein. Keyword1 --- Keyword2 --- Keyword3} % Keywords
\newcommand{\keywordname}{Sleutelwoorden} % Defines the keywords heading name

%---------- Titel, inhoud -----------------------------------------------------

\begin{document}

\flushbottom % Makes all text pages the same height
\maketitle % Print the title and abstract box
\tableofcontents % Print the contents section
\thispagestyle{empty} % Removes page numbering from the first page

%------------------------------------------------------------------------------
% Hoofdtekst
%------------------------------------------------------------------------------

% De hoofdtekst van het voorstel zit in een apart bestand, zodat het makkelijk
% kan opgenomen worden in de bijlagen van de bachelorproef zelf.
%---------- Inleiding ---------------------------------------------------------

\section{Introductie} % The \section*{} command stops section numbering
\label{sec:introductie}

In deze tijd is het bijna onmogelijk om de woorden “cryptocurrency” en “blockchain” nog niet gehoord te hebben. De meeste mensen kunnen waarschijnlijk wel een vage omschrijving geven van deze woorden, maar weinig mensen weten wat ze echt inhouden. Op heel veel sites en diensten kan er al gebruik gemaakt worden van deze betaalmethode. Meer en meer staan de valuta eenheden zoals Amerikaanse dollar, euro, Briste pond,… hun plaats af voor cryptocurrencies. Daarom lijkt het mij passend om hierover een studie te voeren. In dit onderzoek zal er gekeken worden naar hoe winstgevend trading bots daadwerkelijk zijn in vergelijking met menselijke traders, en hoe ze in elkaar zitten. Ook zal er uitgelegd worden wat de termen “cryptocurrency” en “blockchain” nu werkelijk betekenen. Tijdens dit onderzoek zal er een crypto trading bot geprogrammeerd worden, waarna de prestatie vergeleken zal worden met verschillende traders. Bij het traden van crypto, aandelen of grondstoffen is het uitschakelen van emoties een belangrijke factor, ook het snel handelen en inzien van mogelijkheden zijn belangrijke factoren. Omwille van deze redenen wordt er verwacht dat een trading bot in het algemeen betere prestaties zou moeten leveren.


%---------- Stand van zaken ---------------------------------------------------

\section{State-of-the-art}
\label{sec:state-of-the-art}

Cryptogeld is waarde in de vorm van een bedrag in een cryptovaluta, of – met een Engels woord – cryptocurrency. Dit laatste is een soort digitale munteenheid, die vaak gebruikt wordt als alternatief geldsysteem voor de reguliere geldsoorten. 's Werelds bekendste cryptovaluta is de bitcoin. De marktkapitalisatie van cryptocurrency is geraamd op bijna 2.000 miljard dollar begin 2021. \autocite{Wikipedia2022} Dit immens hoge en nog steeds groeiende bedrag, samen met het feit dat heel weinig mensen juist weten wat dit allemaal inhoudt is een reden om hier dieper op in te gaan. Cryptogeld is iets virtueel en zoals de naam al aangeeft komt het woord cryptografie hierin voor. Om hackers tegen te gaan zijn er immens grote getallen die als een soort van muur de cryptocurrency beschermen, het raden van deze getallenreeks is zo goed als onmogelijk. Of zoals \textcite{Nakamoto2010}, de maker van bitcoin, het ooit zei: SHA-256 is heel sterk.  Het is niet zoals de incrementele stap van MD5 naar SHA1. Het kan verschillende decennia’s bestaan, tenzij er een massieve doorbraak plaats vindt. Niet alleen worden cryptocurrencies gebruikt om transacties te doen, ook wordt er in de cryptowereld aan mining en trading gedaan. Mining is het in circulatie brengen van nieuwe munten, door complexe wiskundige formules snel te laten uitvoeren en zo het juiste getal te raden. Een eigenschap aan cryptogeld is dat er een bepaald aantal munten bestaan, en dit waarschijnlijk niet meer aangepast zal worden. Door het minen van cryptocurrencies worden de nog niet gevonden munten aan de blockchain toegevoegd. Eenmaal alle bestaande munten gevonden zijn, is het niet meer mogelijk om aan mining te doen. Het traden is het verhandelen van verschillende crytocurrencies en hopen dat het mogelijk is te verkopen aan een hogere prijs en zo winst te maken. In dit onderzoek zal er een programma geprogrammeerd worden voor het uitvoeren van deze transacties. Vervolgens zal er vergeleken worden of dit lucratiever is dan wanneer mensen deze transacties op zich nemen. Er zijn al verschillende gelijkaardige studies gevoerd, een studie van \textcite{MarliDamiaoAbade2021} concludeert dat robots effectief efficiënter zijn, echter gaat deze studie wel over het traden op de forex markt en gebeurt dit met een zeer geavanceerde robot. Ook wordt er in deze bachelorproef dieper ingegaan op de begrippen cryptocurrency en blockchain. Er bestaan al verschillende trading bots, ook worden er online trading bots aangeboden waar je mits een maandelijkse betaling van gebruik kan maken. Deze zijn vaak heel geavanceerd en lucratief, maar komen ook tegen een aardig prijskaartje. In dit onderzoek zal slechts de basis geprogrammeerd worden.


%---------- Methodologie ------------------------------------------------------
\section{Methodologie}
\label{sec:methodologie}

In de eerste fase zal uitgezocht worden welke manier van traden het beste zal zijn voor het implementeren van de cryptotrading bot. Mogelijke manieren zijn het voorspellen van de koersen aan de hand van voorafgaande grafieken, ook wel trendfollowing genoemd. Dit is een eerder risicovolle manier omdat er geen zekerheid is van wat de toekomst brengt. Een tweede manier is arbitrage, dit is een fenomeen uit de wiskunde. Hierbij wordt er gekeken naar verschillende platformen. Bijvoorbeeld indien platform A een cryptocurrency aan een bepaalde prijs verkoopt die lager is dan de prijs op platform B, kan men de cryptocurrency op platform A kopen en vervolgens verkopen op platform B. Dit zijn niet de enige mogelijkheden, maar het oplijsten van alle manier zou een te lange lijst geven. Vervolgens is de tweede fase het kiezen van de programmeertaal. De derde fase is het effectief implementeren van de trading bot. De vierde fase en laatste fase is het analyseren van de trading bot in vergelijking met echte traders. Wat het verschil in winst/verlies is, de hoeveelheid werk er in gestoken moet worden enzovoort. Hiervoor zal er contact opgenomen worden met traders om zo hun cijfers te vergelijken met die van de trading bot.

%---------- Verwachte resultaten ----------------------------------------------
\section{Verwachte resultaten}
\label{sec:verwachte_resultaten}

De eerste stap zal de meest cruciale stap in het proces zijn, dit zal voor een groot deel bepalen wat de mogelijke winst/verlies zal zijn en het risico dat er mee gepaard zal gaan, en zal dus ook de uitkomst van het onderzoek sterk bepalen. De tweede stap is een vrij eenvoudige stap, en zal waarschijnlijk de meest gebruikte taal worden. Vervolgens is er de derde stap, deze zal hoogstwaarschijnlijk het meeste werk kosten. Hierbij zal er heel veel opzoekwerk gedaan moeten worden, omdat dit een eerder onbekend onderwerp is. Er wordt verwacht dat dit meer dan mogelijk is om klaar te hebben in een periode van ongeveer twee maanden. De laatste fase zal niet altijd van een leien dakje lopen omdat desondanks de marktwaarde van cryptocurrencies al ruim boven de 2000 miljard dollar ligt, er wordt geschat dat in begin 2021 er slechts 3.9\% van de hele bevolking cryptocurrencies bezit. \autocite{Triple2021} Hierdoor zal het moeilijker zijn om iemand te vinden die zijn bevindingen wil delen. Niet te min is er een heel actieve community rond cryptogeld. De verwachtte resultaten zijn hier dat trading bots het beter zouden doen dan menselijke traders.

%---------- Verwachte conclusies ----------------------------------------------
\section{Verwachte conclusies}
\label{sec:verwachte_conclusies}

Trading bots zullen waarschijnlijk een voordeel hebben op beginnende mensen, maar eenmaal er vergeleken wordt met een expert in het verhandelen van cryptogeld wordt er verwacht dat traders een voordeel zullen hebben. Ook zal het misschien winstgevend zijn, maar hier zal ongetwijfeld veel werk en moeite tegenover staan. Een programma waar meerdere factoren in opgenomen worden zal waarschijnlijk ook beter presteren.



%------------------------------------------------------------------------------
% Referentielijst
%------------------------------------------------------------------------------
% TODO: de gerefereerde werken moeten in BibTeX-bestand ``voorstel.bib''
% voorkomen. Gebruik JabRef om je bibliografie bij te houden en vergeet niet
% om compatibiliteit met Biber/BibLaTeX aan te zetten (File > Switch to
% BibLaTeX mode)

\phantomsection
\printbibliography[heading=bibintoc]

\end{document}
