%%=============================================================================
%% Inleiding
%%=============================================================================

\chapter{\IfLanguageName{dutch}{Inleiding}{Introduction}}
\label{ch:inleiding}

De inleiding moet de lezer net genoeg informatie verschaffen om het onderwerp te begrijpen en in te zien waarom de onderzoeksvraag de moeite waard is om te onderzoeken. In de inleiding ga je literatuurverwijzingen beperken, zodat de tekst vlot leesbaar blijft. Je kan de inleiding verder onderverdelen in secties als dit de tekst verduidelijkt. Zaken die aan bod kunnen komen in de inleiding

\begin{itemize}
  \item context, achtergrond
  \item afbakenen van het onderwerp
  \item verantwoording van het onderwerp, methodologie
  \item probleemstelling
  \item onderzoeksdoelstelling
  \item onderzoeksvraag
  \item \ldots
\end{itemize}

\section{\IfLanguageName{dutch}{Probleemstelling}{Problem Statement}}
\label{sec:probleemstelling}

Uit je probleemstelling moet duidelijk zijn dat je onderzoek een meerwaarde heeft voor een concrete doelgroep. De doelgroep moet goed gedefinieerd en afgelijnd zijn. Doelgroepen als ``bedrijven,'' ``KMO's,'' systeembeheerders, enz.~zijn nog te vaag. Als je een lijstje kan maken van de personen/organisaties die een meerwaarde zullen vinden in deze bachelorproef (dit is eigenlijk je steekproefkader), dan is dat een indicatie dat de doelgroep goed gedefinieerd is. Dit kan een enkel bedrijf zijn of zelfs één persoon (je co-promotor/opdrachtgever).

\section{\IfLanguageName{dutch}{Onderzoeksvraag}{Research question}}
\label{sec:onderzoeksvraag}

Wees zo concreet mogelijk bij het formuleren van je onderzoeksvraag. Een onderzoeksvraag is trouwens iets waar nog niemand op dit moment een antwoord heeft (voor zover je kan nagaan). Het opzoeken van bestaande informatie (bv. ``welke tools bestaan er voor deze toepassing?'') is dus geen onderzoeksvraag. Je kan de onderzoeksvraag verder specifiëren in deelvragen. Bv.~als je onderzoek gaat over performantiemetingen, dan

\section{\IfLanguageName{dutch}{Onderzoeksdoelstelling}{Research objective}}
\label{sec:onderzoeksdoelstelling}

Wat is het beoogde resultaat van je bachelorproef? Wat zijn de criteria voor succes? Beschrijf die zo concreet mogelijk. Gaat het bv. om een proof-of-concept, een prototype, een verslag met aanbevelingen, een vergelijkende studie, enz.

\section{\IfLanguageName{dutch}{Opzet van deze bachelorproef}{Structure of this bachelor thesis}}
\label{sec:opzet-bachelorproef}

% Het is gebruikelijk aan het einde van de inleiding een overzicht te
% geven van de opbouw van de rest van de tekst. Deze sectie bevat al een aanzet
% die je kan aanvullen/aanpassen in functie van je eigen tekst.

De rest van deze bachelorproef is als volgt opgebouwd:

In Hoofdstuk~\ref{ch:stand-van-zaken} wordt een overzicht gegeven van de stand van zaken binnen het onderzoeksdomein, op basis van een literatuurstudie.

In Hoofdstuk~\ref{ch:methodologie} wordt de methodologie toegelicht en worden de gebruikte onderzoekstechnieken besproken om een antwoord te kunnen formuleren op de onderzoeksvragen.

% TODO: Vul hier aan voor je eigen hoofstukken, één of twee zinnen per hoofdstuk

In Hoofdstuk~\ref{ch:conclusie}, tenslotte, wordt de conclusie gegeven en een antwoord geformuleerd op de onderzoeksvragen. Daarbij wordt ook een aanzet gegeven voor toekomstig onderzoek binnen dit domein.











Cryptocurrencies, het is voor sommigen een vaag en ongekend begrip. Terwijl het voor anderen een woord is dat niet meer uit hun leven weg te denken is. Het ontstaan van de eerste cryptocurrency, namelijk bitcoin begin januari 2009 \autocite{}, bracht tal van nieuwe mogelijkheden met zich mee om geld te verdienen. Zaken zoals
\underline{crypto mining} en \underline{crypto trading} kwamen naar buiten als een alternatieve vorm van investering. Naast de \underline{forex} en effecten markten werd het meer dan normaal om een deel van je investeringsportefeuille in cryptocurrencies te investeren. Echter wordt aangeraden slechts een klein deel hiervoor aan de kant te zetten. De reden hiervoor is dat de crypto markt heel \underline{volatiel} is en omdat er geen zekerheid is of cryptocurrencies ook daadwerkelijk overal aanzien gaan worden als een alternatieve munteenheid.

In deze bachelorproef zal er niet aan crypto mining, maar aan crypto trading worden gedaan. Er zal nagegaan worden of er een beter rendement kan behaald worden dan de simpele buy-and-hold stragtegie. Pas indien dit het geval is, kan er gesproken worden van een geslaagd experiment. Aan de hand van gesofisticeerde wiskunde formules die op basis van de voorbije prijzen, trading volumes enzovoort de koers proberen te voorspellen zullen er munten aangekocht of verkocht worden. Dit zal echter gebeuren met beperkte middelen en binnen een beperkte tijd. Men kan dan kiezen om long of short te gaan, wat inzetten op respectievelijk een stijging of daling van de valuta inhoudt. Dit zal allemaal geautomatiseerd worden aan de hand van code die met een \underline{API} van een broker zal communiceren en de orders doorgeven. De volatiliteit dat eerder vernoemd werd is de reden waarom cryptogeld zo populair is onder traders, ook het lage startbedrag speelt een grote rol. Waar je in de effectenmarkt grote sommen geld op tafel moet leggen om nog maar te mogen traden is dit met cryptocurrencies helemaal anders. Hier kan elke persoon beginnen met het verhandelen van cryptomunten.





