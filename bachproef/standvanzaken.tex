\chapter{\IfLanguageName{dutch}{Stand van zaken}{State of the art}}
\label{ch:stand-van-zaken}

% Tip: Begin elk hoofdstuk met een paragraaf inleiding die beschrijft hoe
% dit hoofdstuk past binnen het geheel van de bachelorproef. Geef in het
% bijzonder aan wat de link is met het vorige en volgende hoofdstuk.

% Pas na deze inleidende paragraaf komt de eerste sectiehoofding.

%Dit hoofdstuk bevat je literatuurstudie. De inhoud gaat verder op de inleiding, maar zal het onderwerp van de bachelorproef *diepgaand* uitspitten. De bedoeling is dat de lezer na lezing van dit hoofdstuk helemaal op de hoogte is van de huidige stand van zaken (state-of-the-art) in het onderzoeksdomein. Iemand die niet vertrouwd is met het onderwerp, weet nu voldoende om de rest van het verhaal te kunnen volgen, zonder dat die er nog andere informatie moet over opzoeken.
%
%Je verwijst bij elke bewering die je doet, vakterm die je introduceert, enz. naar je bronnen. In \LaTeX{} kan dat met het commando \texttt{$\backslash${textcite\{\}}} of \texttt{$\backslash${autocite\{\}}}. Als argument van het commando geef je de ``sleutel'' van een ``record'' in een bibliografische databank in het Bib\LaTeX{}-formaat (een tekstbestand). Als je expliciet naar de auteur verwijst in de zin, gebruik je \texttt{$\backslash${}textcite\{\}}.
%Soms wil je de auteur niet expliciet vernoemen, dan gebruik je \texttt{$\backslash${}autocite\{\}}. In de volgende paragraaf een voorbeeld van elk.








\section{De geschiedenis en werking van cryptocurrencies en blockchains}

Cryptocurrencies zijn onstaan met het idee dat er iets mis was met de huidige manier van betalen. Zo kunnen er allerlei zaken mis gaan wanneer je een betaling uitvoert via een bank of derde partij. Ook is het vervalsen van cryptomunten minder een probleem dan bij \underline{fiat currencies} zoals euro, Amerikaanse doller enzovoort. Na het onstaan van bitcoin, de eerste cryptomunt volgden de andere cryptomunten snel. \textbf{Tabel van de datums dat de bekendste crypto munten uitgekomen zijn?} Maar het zijn niet noodzakelijk de oudste munten die het populairst zijn. Zo is Ethereum een munt die veel wordt verhandeld binnen de crypto community pas onstaan in 2015 \autocite{}. Ethereum wordt ook veel gebruikt voor het verhandelen van NFT's. Een NFT is een Non-Fungible Token: een niet- inwisselbaar, onvervangbaar digitaal eigendomscertificaat. Dit certificaat kun je digitaal creëren en koppelen aan een digitaal object (zoals een afbeelding) en dat registreer je vervolgens in een blockchain. Deze registratie is daarnaast ook gekoppeld aan jouw eigen digitale cryptoportemonnee. Mensen kunnen zien dat jij de eigenaar bent, maar niemand kan die registratie wijzigen. \autocite{Duursma2021}. In de uitleg over NFT's spreek \textcite{Duursma2021} over blockchains. Deze kunnen gezien worden als het ware als openbare databases waarin allerlei transacties worden opgeslagen. Deze databases worden voor zeer veel verschillende zaken gebruikt. Dit kan gaan van het verhandelen van eigendomsbewijzen, contracten enzovoort, maar ze worden ook gebruikt om cryptocurrencies op te verhandelen.

\section{Manieren van algorithmic traden en de mogelijke hindernissen}



\section{Wiskundige formules voor het berekenen van de koersen}



\section{Hoe de wereld crypto beïnvloed en hoe crypto de wereld beïnvloed}



\section{Mogelijke brokers, hun voor-en nadelen}