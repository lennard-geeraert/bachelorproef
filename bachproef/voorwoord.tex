%%=============================================================================
%% Voorwoord
%%=============================================================================

\chapter*{\IfLanguageName{dutch}{Woord vooraf}{Preface}}
\label{ch:voorwoord}

%% TODO:
%% Het voorwoord is het enige deel van de bachelorproef waar je vanuit je
%% eigen standpunt (``ik-vorm'') mag schrijven. Je kan hier bv. motiveren
%% waarom jij het onderwerp wil bespreken.
%% Vergeet ook niet te bedanken wie je geholpen/gesteund/... heeft

In deze tijd is het bijna onmogelijk om de woorden “cryptocurrency” en “blockchain” nog niet gehoord te hebben. De meeste mensen kunnen waarschijnlijk wel een vage omschrijving geven van deze woorden, maar weinig mensen weten wat ze echt inhouden. Op heel veel sites en diensten kan er al gebruik gemaakt worden van deze betaalmethode. Meer en meer staan de valuta eenheden zoals Amerikaanse dollar, euro, Briste pond,… hun plaats af voor cryptocurrencies. Daarom lijkt het mij passend om hierover een studie te voeren. In dit onderzoek zal er gekeken worden naar hoe winstgevend trading bots daadwerkelijk zijn, en hoe ze in elkaar zitten.

Ook zal er uitgelegd worden wat de termen “cryptocurrency” en “blockchain” nu werkelijk betekenen. Tijdens dit onderzoek zal er een crypto trading bot geprogrammeerd worden, waarna zal nagegaan worden hoe winstgevend of niet winstgevend deze bot was in vergeljiking met de moeite die erin werd gestoken. Bij het traden van crypto, aandelen of grondstoffen... is het uitschakelen van emoties een belangrijke factor, ook het snel handelen en inzien van mogelijkheden is belangrijk. Omwille van deze redenen wordt er verwacht dat een trading bot in het algemeen betere prestaties zou moeten leveren.