
\documentclass[a4paper]{bachproef-tin}

\usepackage{hogent-thesis-titlepage} % Titelpagina conform aan HOGENT huisstijl

% packages van video mr. Van Vreckem
% , amsfonts, amsmath, amssymb, babel, eurosym, graphicx, hyperref, inputenc, listings, multirow, rotating, lipsum, booktabs, listings
% [dutch, pdftex, bookmarks=true, utf8]

%%---------- Documenteigenschappen ---------------------------------------------

\title{De rentabiliteit van een zelfgeprogrammeerde crypto trading bot}

\author{Lennard Geeraert}

\promotor{Giselle Vercauteren}

\copromotor{Geert De Vuyst}

\instelling{---}

\academiejaar{2021-2022}

% Examenperiode
%  - 1e semester = 1e examenperiode => 1
%  - 2e semester = 2e examenperiode => 2
%  - tweede zit  = 3e examenperiode => 3
\examenperiode{2}

%===============================================================================
% Inhoud document
%===============================================================================

\begin{document}

%---------- Taalselectie -------------------------------------------------------
% Als je je bachelorproef in het Engels schrijft, haal dan onderstaande regel
% uit commentaar. Let op: de tekst op de voorkaft blijft in het Nederlands, en
% dat is ook de bedoeling!

%\selectlanguage{english}

%---------- Titelblad ----------------------------------------------------------
\inserttitlepage

%---------- Samenvatting, voorwoord --------------------------------------------
\usechapterimagefalse
%%=============================================================================
%% Voorwoord
%%=============================================================================

\chapter*{\IfLanguageName{dutch}{Woord vooraf}{Preface}}
\label{ch:voorwoord}

%% TODO:
%% Het voorwoord is het enige deel van de bachelorproef waar je vanuit je
%% eigen standpunt (``ik-vorm'') mag schrijven. Je kan hier bv. motiveren
%% waarom jij het onderwerp wil bespreken.
%% Vergeet ook niet te bedanken wie je geholpen/gesteund/... heeft

In deze tijd is het bijna onmogelijk om de woorden “cryptocurrency” en “blockchain” nog niet gehoord te hebben. De meeste mensen kunnen waarschijnlijk wel een vage omschrijving geven van deze woorden, maar weinig mensen weten wat ze echt inhouden. Op heel veel sites en diensten kan er al gebruik gemaakt worden van deze betaalmethode. Meer en meer staan de valuta eenheden zoals Amerikaanse dollar, euro, Briste pond,… hun plaats af voor cryptocurrencies. Daarom lijkt het mij passend om hierover een studie te voeren. In dit onderzoek zal er gekeken worden naar hoe winstgevend trading bots daadwerkelijk zijn, en hoe ze in elkaar zitten.

Ook zal er uitgelegd worden wat de termen “cryptocurrency” en “blockchain” nu werkelijk betekenen. Tijdens dit onderzoek zal er een crypto trading bot geprogrammeerd worden, waarna zal nagegaan worden hoe winstgevend of niet winstgevend deze bot was in vergeljiking met de moeite die erin werd gestoken. Bij het traden van crypto, aandelen of grondstoffen... is het uitschakelen van emoties een belangrijke factor, ook het snel handelen en inzien van mogelijkheden is belangrijk. Omwille van deze redenen wordt er verwacht dat een trading bot in het algemeen betere prestaties zou moeten leveren.
%%=============================================================================
%% Samenvatting
%%=============================================================================

% TODO: De "abstract" of samenvatting is een kernachtige (~ 1 blz. voor een
% thesis) synthese van het document.
%
% Deze aspecten moeten zeker aan bod komen:
% - Context: waarom is dit werk belangrijk?
% - Nood: waarom moest dit onderzocht worden?
% - Taak: wat heb je precies gedaan?
% - Object: wat staat in dit document geschreven?
% - Resultaat: wat was het resultaat?
% - Conclusie: wat is/zijn de belangrijkste conclusie(s)?
% - Perspectief: blijven er nog vragen open die in de toekomst nog kunnen
%    onderzocht worden? Wat is een mogelijk vervolg voor jouw onderzoek?
%
% LET OP! Een samenvatting is GEEN voorwoord!

%%---------- Nederlandse samenvatting -----------------------------------------
%
% TODO: Als je je bachelorproef in het Engels schrijft, moet je eerst een
% Nederlandse samenvatting invoegen. Haal daarvoor onderstaande code uit
% commentaar.
% Wie zijn bachelorproef in het Nederlands schrijft, kan dit negeren, de inhoud
% wordt niet in het document ingevoegd.

\IfLanguageName{english}{%
\selectlanguage{dutch}
\chapter*{Samenvatting}
\lipsum[1-4]
\selectlanguage{english}
}{}

%%---------- Samenvatting -----------------------------------------------------
% De samenvatting in de hoofdtaal van het document

\chapter*{\IfLanguageName{dutch}{Samenvatting}{Abstract}}

\lipsum[1-4]


%---------- Inhoudstafel -------------------------------------------------------
\pagestyle{empty} % Geen hoofding
\tableofcontents  % Voeg de inhoudstafel toe
% \cleardoublepage  % Zorg dat volgende hoofstuk op een oneven pagina begint
\pagestyle{fancy} % Zet hoofding opnieuw aan

%---------- Lijst figuren, afkortingen, ... ------------------------------------

% Indien gewenst kan je hier een lijst van figuren/tabellen opgeven. Geef in
% dat geval je figuren/tabellen altijd een korte beschrijving:
%
%  \caption[korte beschrijving]{uitgebreide beschrijving}
%
% De korte beschrijving wordt gebruikt voor deze lijst, de uitgebreide staat bij
% de figuur of tabel zelf.

% \listoffigures
% \listoftables
% \lstlistoflistings

% Als je een lijst van afkortingen of termen wil toevoegen, dan hoort die
% hier thuis. Gebruik bijvoorbeeld de ``glossaries'' package.
% https://www.overleaf.com/learn/latex/Glossaries

%---------- Kern ---------------------------------------------------------------

% De eerste hoofdstukken van een bachelorproef zijn meestal een inleiding op
% het onderwerp, literatuurstudie en verantwoording methodologie.
% Aarzel niet om een meer beschrijvende titel aan deze hoofstukken te geven of
% om bijvoorbeeld de inleiding en/of stand van zaken over meerdere hoofdstukken
% te verspreiden!

%%=============================================================================
%% Inleiding
%%=============================================================================

\chapter{\IfLanguageName{dutch}{Inleiding}{Introduction}}
\label{ch:inleiding}

De inleiding moet de lezer net genoeg informatie verschaffen om het onderwerp te begrijpen en in te zien waarom de onderzoeksvraag de moeite waard is om te onderzoeken. In de inleiding ga je literatuurverwijzingen beperken, zodat de tekst vlot leesbaar blijft. Je kan de inleiding verder onderverdelen in secties als dit de tekst verduidelijkt. Zaken die aan bod kunnen komen in de inleiding

\begin{itemize}
  \item context, achtergrond
  \item afbakenen van het onderwerp
  \item verantwoording van het onderwerp, methodologie
  \item probleemstelling
  \item onderzoeksdoelstelling
  \item onderzoeksvraag
  \item \ldots
\end{itemize}

\section{\IfLanguageName{dutch}{Probleemstelling}{Problem Statement}}
\label{sec:probleemstelling}

Uit je probleemstelling moet duidelijk zijn dat je onderzoek een meerwaarde heeft voor een concrete doelgroep. De doelgroep moet goed gedefinieerd en afgelijnd zijn. Doelgroepen als ``bedrijven,'' ``KMO's,'' systeembeheerders, enz.~zijn nog te vaag. Als je een lijstje kan maken van de personen/organisaties die een meerwaarde zullen vinden in deze bachelorproef (dit is eigenlijk je steekproefkader), dan is dat een indicatie dat de doelgroep goed gedefinieerd is. Dit kan een enkel bedrijf zijn of zelfs één persoon (je co-promotor/opdrachtgever).

\section{\IfLanguageName{dutch}{Onderzoeksvraag}{Research question}}
\label{sec:onderzoeksvraag}

Wees zo concreet mogelijk bij het formuleren van je onderzoeksvraag. Een onderzoeksvraag is trouwens iets waar nog niemand op dit moment een antwoord heeft (voor zover je kan nagaan). Het opzoeken van bestaande informatie (bv. ``welke tools bestaan er voor deze toepassing?'') is dus geen onderzoeksvraag. Je kan de onderzoeksvraag verder specifiëren in deelvragen. Bv.~als je onderzoek gaat over performantiemetingen, dan

\section{\IfLanguageName{dutch}{Onderzoeksdoelstelling}{Research objective}}
\label{sec:onderzoeksdoelstelling}

Wat is het beoogde resultaat van je bachelorproef? Wat zijn de criteria voor succes? Beschrijf die zo concreet mogelijk. Gaat het bv. om een proof-of-concept, een prototype, een verslag met aanbevelingen, een vergelijkende studie, enz.

\section{\IfLanguageName{dutch}{Opzet van deze bachelorproef}{Structure of this bachelor thesis}}
\label{sec:opzet-bachelorproef}

% Het is gebruikelijk aan het einde van de inleiding een overzicht te
% geven van de opbouw van de rest van de tekst. Deze sectie bevat al een aanzet
% die je kan aanvullen/aanpassen in functie van je eigen tekst.

De rest van deze bachelorproef is als volgt opgebouwd:

In Hoofdstuk~\ref{ch:stand-van-zaken} wordt een overzicht gegeven van de stand van zaken binnen het onderzoeksdomein, op basis van een literatuurstudie.

In Hoofdstuk~\ref{ch:methodologie} wordt de methodologie toegelicht en worden de gebruikte onderzoekstechnieken besproken om een antwoord te kunnen formuleren op de onderzoeksvragen.

% TODO: Vul hier aan voor je eigen hoofstukken, één of twee zinnen per hoofdstuk

In Hoofdstuk~\ref{ch:conclusie}, tenslotte, wordt de conclusie gegeven en een antwoord geformuleerd op de onderzoeksvragen. Daarbij wordt ook een aanzet gegeven voor toekomstig onderzoek binnen dit domein.











Cryptocurrencies, het is voor sommigen een vaag en ongekend begrip. Terwijl het voor anderen een woord is dat niet meer uit hun leven weg te denken is. Het ontstaan van de eerste cryptocurrency, namelijk bitcoin begin januari 2009 \autocite{}, bracht tal van nieuwe mogelijkheden met zich mee om geld te verdienen. Zaken zoals
\underline{crypto mining} en \underline{crypto trading} kwamen naar buiten als een alternatieve vorm van investering. Naast de \underline{forex} en effecten markten werd het meer dan normaal om een deel van je investeringsportefeuille in cryptocurrencies te investeren. Echter wordt aangeraden slechts een klein deel hiervoor aan de kant te zetten. De reden hiervoor is dat de crypto markt heel \underline{volatiel} is en omdat er geen zekerheid is of cryptocurrencies ook daadwerkelijk overal aanzien gaan worden als een alternatieve munteenheid.

In deze bachelorproef zal er niet aan crypto mining, maar aan crypto trading worden gedaan. Er zal nagegaan worden of er een beter rendement kan behaald worden dan de simpele buy and hold stragtegie. Pas indien dit het geval is, kan er gesproken worden van een geslaagd experiment. Aan de hand van gesofisticeerde wiskunde formules die op basis van de voorbije prijzen, trading volumes enzovoort de koers proberen te voorspellen zullen er munten aangekocht of verkocht worden. Dit zal echter gebeuren met beperkte middelen en binnen een beperkte tijd. Men kan dan kiezen om long of short te gaan, wat inzetten op respectievelijk een stijging of daling van de valuta inhoudt. Dit zal allemaal geautomatiseerd worden aan de hand van code die met een \underline{API} van een broker zal communiceren en de orders doorgeven. De volatiliteit dat eerder vernoemd werd is de reden waarom cryptogeld zo populair is onder traders, ook het lage startbedrag speelt een grote rol. Waar je in de effectenmarkt grote sommen geld op tafel moet leggen om nog maar te mogen traden is dit met cryptocurrencies helemaal anders. Hier kan elke persoon beginnen met het verhandelen van cryptomunten.






\chapter{\IfLanguageName{dutch}{Stand van zaken}{State of the art}}
\label{ch:stand-van-zaken}

% Tip: Begin elk hoofdstuk met een paragraaf inleiding die beschrijft hoe
% dit hoofdstuk past binnen het geheel van de bachelorproef. Geef in het
% bijzonder aan wat de link is met het vorige en volgende hoofdstuk.

% Pas na deze inleidende paragraaf komt de eerste sectiehoofding.

Dit hoofdstuk bevat je literatuurstudie. De inhoud gaat verder op de inleiding, maar zal het onderwerp van de bachelorproef *diepgaand* uitspitten. De bedoeling is dat de lezer na lezing van dit hoofdstuk helemaal op de hoogte is van de huidige stand van zaken (state-of-the-art) in het onderzoeksdomein. Iemand die niet vertrouwd is met het onderwerp, weet nu voldoende om de rest van het verhaal te kunnen volgen, zonder dat die er nog andere informatie moet over opzoeken \autocite{Pollefliet2011}.

Je verwijst bij elke bewering die je doet, vakterm die je introduceert, enz. naar je bronnen. In \LaTeX{} kan dat met het commando \texttt{$\backslash${textcite\{\}}} of \texttt{$\backslash${autocite\{\}}}. Als argument van het commando geef je de ``sleutel'' van een ``record'' in een bibliografische databank in het Bib\LaTeX{}-formaat (een tekstbestand). Als je expliciet naar de auteur verwijst in de zin, gebruik je \texttt{$\backslash${}textcite\{\}}.
Soms wil je de auteur niet expliciet vernoemen, dan gebruik je \texttt{$\backslash${}autocite\{\}}. In de volgende paragraaf een voorbeeld van elk.

\textcite{Knuth1998} schreef een van de standaardwerken over sorteer- en zoekalgoritmen. Experten zijn het erover eens dat cloud computing een interessante opportuniteit vormen, zowel voor gebruikers als voor dienstverleners op vlak van informatietechnologie~\autocite{Creeger2009}.

\lipsum[7-20]

%%=============================================================================
%% Methodologie
%%=============================================================================

\chapter{\IfLanguageName{dutch}{Methodologie}{Methodology}}
\label{ch:methodologie}

%% Hoe ben je te werk gegaan? Verdeel je onderzoek in grote fasen, en
%% licht in elke fase toe welke stappen je gevolgd hebt. Verantwoord waarom je
%% op deze manier te werk gegaan bent. Je moet kunnen aantonen dat je de best
%% mogelijke manier toegepast hebt om een antwoord te vinden op de
%% onderzoeksvraag.

%\lipsum[21-25]



% Voeg hier je eigen hoofdstukken toe die de ``corpus'' van je bachelorproef
% vormen. De structuur en titels hangen af van je eigen onderzoek. Je kan bv.
% elke fase in je onderzoek in een apart hoofdstuk bespreken.

%\input{...}
%\input{...}
%...

%%=============================================================================
%% Conclusie
%%=============================================================================

\chapter{Conclusie}
\label{ch:conclusie}

% TODO: Trek een duidelijke conclusie, in de vorm van een antwoord op de
% onderzoeksvra(a)g(en). Wat was jouw bijdrage aan het onderzoeksdomein en
% hoe biedt dit meerwaarde aan het vakgebied/doelgroep? 
% Reflecteer kritisch over het resultaat. In Engelse teksten wordt deze sectie
% ``Discussion'' genoemd. Had je deze uitkomst verwacht? Zijn er zaken die nog
% niet duidelijk zijn?
% Heeft het onderzoek geleid tot nieuwe vragen die uitnodigen tot verder 
%onderzoek?

\lipsum[76-80]



%%=============================================================================
%% Bijlagen
%%=============================================================================

\appendix
\renewcommand{\chaptername}{Appendix}

%%---------- Onderzoeksvoorstel -----------------------------------------------

\chapter{Onderzoeksvoorstel}

%---------- Inleiding ---------------------------------------------------------

\section{Introductie} % The \section*{} command stops section numbering
\label{sec:introductie}

In deze tijd is het bijna onmogelijk om de woorden “cryptocurrency” en “blockchain” nog niet gehoord te hebben. De meeste mensen kunnen waarschijnlijk wel een vage omschrijving geven van deze woorden, maar weinig mensen weten wat ze echt inhouden. Op heel veel sites en diensten kan er al gebruik gemaakt worden van deze betaalmethode. Meer en meer staan de valuta eenheden zoals Amerikaanse dollar, euro, Briste pond,… hun plaats af voor cryptocurrencies. Daarom lijkt het mij passend om hierover een studie te voeren. In dit onderzoek zal er gekeken worden naar hoe winstgevend trading bots daadwerkelijk zijn in vergelijking met menselijke traders, en hoe ze in elkaar zitten. Ook zal er uitgelegd worden wat de termen “cryptocurrency” en “blockchain” nu werkelijk betekenen. Tijdens dit onderzoek zal er een crypto trading bot geprogrammeerd worden, waarna de prestatie vergeleken zal worden met verschillende traders. Bij het traden van crypto, aandelen of grondstoffen is het uitschakelen van emoties een belangrijke factor, ook het snel handelen en inzien van mogelijkheden zijn belangrijke factoren. Omwille van deze redenen wordt er verwacht dat een trading bot in het algemeen betere prestaties zou moeten leveren.


%---------- Stand van zaken ---------------------------------------------------

\section{State-of-the-art}
\label{sec:state-of-the-art}

Cryptogeld is waarde in de vorm van een bedrag in een cryptovaluta, of – met een Engels woord – cryptocurrency. Dit laatste is een soort digitale munteenheid, die vaak gebruikt wordt als alternatief geldsysteem voor de reguliere geldsoorten. 's Werelds bekendste cryptovaluta is de bitcoin. De marktkapitalisatie van cryptocurrency is geraamd op bijna 2.000 miljard dollar begin 2021. \autocite{Wikipedia2022} Dit immens hoge en nog steeds groeiende bedrag, samen met het feit dat heel weinig mensen juist weten wat dit allemaal inhoudt is een reden om hier dieper op in te gaan. Cryptogeld is iets virtueel en zoals de naam al aangeeft komt het woord cryptografie hierin voor. Om hackers tegen te gaan zijn er immens grote getallen die als een soort van muur de cryptocurrency beschermen, het raden van deze getallenreeks is zo goed als onmogelijk. Of zoals \textcite{Nakamoto2010}, de maker van bitcoin, het ooit zei: SHA-256 is heel sterk.  Het is niet zoals de incrementele stap van MD5 naar SHA1. Het kan verschillende decennia’s bestaan, tenzij er een massieve doorbraak plaats vindt. Niet alleen worden cryptocurrencies gebruikt om transacties te doen, ook wordt er in de cryptowereld aan mining en trading gedaan. Mining is het in circulatie brengen van nieuwe munten, door complexe wiskundige formules snel te laten uitvoeren en zo het juiste getal te raden. Een eigenschap aan cryptogeld is dat er een bepaald aantal munten bestaan, en dit waarschijnlijk niet meer aangepast zal worden. Door het minen van cryptocurrencies worden de nog niet gevonden munten aan de blockchain toegevoegd. Eenmaal alle bestaande munten gevonden zijn, is het niet meer mogelijk om aan mining te doen. Het traden is het verhandelen van verschillende crytocurrencies en hopen dat het mogelijk is te verkopen aan een hogere prijs en zo winst te maken. In dit onderzoek zal er een programma geprogrammeerd worden voor het uitvoeren van deze transacties. Vervolgens zal er vergeleken worden of dit lucratiever is dan wanneer mensen deze transacties op zich nemen. Er zijn al verschillende gelijkaardige studies gevoerd, een studie van \textcite{MarliDamiaoAbade2021} concludeert dat robots effectief efficiënter zijn, echter gaat deze studie wel over het traden op de forex markt en gebeurt dit met een zeer geavanceerde robot. Ook wordt er in deze bachelorproef dieper ingegaan op de begrippen cryptocurrency en blockchain. Er bestaan al verschillende trading bots, ook worden er online trading bots aangeboden waar je mits een maandelijkse betaling van gebruik kan maken. Deze zijn vaak heel geavanceerd en lucratief, maar komen ook tegen een aardig prijskaartje. In dit onderzoek zal slechts de basis geprogrammeerd worden.


%---------- Methodologie ------------------------------------------------------
\section{Methodologie}
\label{sec:methodologie}

In de eerste fase zal uitgezocht worden welke manier van traden het beste zal zijn voor het implementeren van de cryptotrading bot. Mogelijke manieren zijn het voorspellen van de koersen aan de hand van voorafgaande grafieken, ook wel trendfollowing genoemd. Dit is een eerder risicovolle manier omdat er geen zekerheid is van wat de toekomst brengt. Een tweede manier is arbitrage, dit is een fenomeen uit de wiskunde. Hierbij wordt er gekeken naar verschillende platformen. Bijvoorbeeld indien platform A een cryptocurrency aan een bepaalde prijs verkoopt die lager is dan de prijs op platform B, kan men de cryptocurrency op platform A kopen en vervolgens verkopen op platform B. Dit zijn niet de enige mogelijkheden, maar het oplijsten van alle manier zou een te lange lijst geven. Vervolgens is de tweede fase het kiezen van de programmeertaal. De derde fase is het effectief implementeren van de trading bot. De vierde fase en laatste fase is het analyseren van de trading bot in vergelijking met echte traders. Wat het verschil in winst/verlies is, de hoeveelheid werk er in gestoken moet worden enzovoort. Hiervoor zal er contact opgenomen worden met traders om zo hun cijfers te vergelijken met die van de trading bot.

%---------- Verwachte resultaten ----------------------------------------------
\section{Verwachte resultaten}
\label{sec:verwachte_resultaten}

De eerste stap zal de meest cruciale stap in het proces zijn, dit zal voor een groot deel bepalen wat de mogelijke winst/verlies zal zijn en het risico dat er mee gepaard zal gaan, en zal dus ook de uitkomst van het onderzoek sterk bepalen. De tweede stap is een vrij eenvoudige stap, en zal waarschijnlijk de meest gebruikte taal worden. Vervolgens is er de derde stap, deze zal hoogstwaarschijnlijk het meeste werk kosten. Hierbij zal er heel veel opzoekwerk gedaan moeten worden, omdat dit een eerder onbekend onderwerp is. Er wordt verwacht dat dit meer dan mogelijk is om klaar te hebben in een periode van ongeveer twee maanden. De laatste fase zal niet altijd van een leien dakje lopen omdat desondanks de marktwaarde van cryptocurrencies al ruim boven de 2000 miljard dollar ligt, er wordt geschat dat in begin 2021 er slechts 3.9\% van de hele bevolking cryptocurrencies bezit. \autocite{Triple2021} Hierdoor zal het moeilijker zijn om iemand te vinden die zijn bevindingen wil delen. Niet te min is er een heel actieve community rond cryptogeld. De verwachtte resultaten zijn hier dat trading bots het beter zouden doen dan menselijke traders.

%---------- Verwachte conclusies ----------------------------------------------
\section{Verwachte conclusies}
\label{sec:verwachte_conclusies}

Trading bots zullen waarschijnlijk een voordeel hebben op beginnende mensen, maar eenmaal er vergeleken wordt met een expert in het verhandelen van cryptogeld wordt er verwacht dat traders een voordeel zullen hebben. Ook zal het misschien winstgevend zijn, maar hier zal ongetwijfeld veel werk en moeite tegenover staan. Een programma waar meerdere factoren in opgenomen worden zal waarschijnlijk ook beter presteren.



%%---------- Andere bijlagen --------------------------------------------------
%\input{...}

%%---------- Referentielijst --------------------------------------------------

\printbibliography[heading=bibintoc]

\end{document}
